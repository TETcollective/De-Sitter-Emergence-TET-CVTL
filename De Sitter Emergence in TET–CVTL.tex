\documentclass[11pt,a4paper]{article}
\usepackage[utf8]{inputenc}
\usepackage[T1]{fontenc}
\usepackage{amsmath,amssymb,amsfonts}
\usepackage{graphicx}
\usepackage{caption}
\usepackage{subcaption}
\usepackage{hyperref}
\usepackage{geometry}
\usepackage{tikz}
\usepackage{float}
\geometry{margin=1in}

\hypersetup{
    colorlinks=true,
    linkcolor=blue,
    citecolor=blue,
    urlcolor=blue,
}

\title{De Sitter Emergence in the TET--CVTL Framework: From Primordial Trefoil Saturation to Exponential Cosmic Expansion}
\author{Simon Soliman \\ Independent Researcher, TET Collective, Rome, Italy \\ tetcollective@proton.me}
\date{January 2026}

\begin{document}

\maketitle

\begin{abstract}
This preprint demonstrates that the fully saturated state of the Topological-Entropic Theory with Conformal Vacuum Tensor Lattice (TET--CVTL) naturally emerges as a de Sitter spacetime. The primordial three-leaf clover (trefoil) knot lattice, when taken to maximal multi-knot saturation (Lk=100\%), produces constant positive curvature, maximal entanglement entropy, and exponential expansion driven by an emergent cosmological constant $\Lambda$. This provides a parameter-free derivation of de Sitter geometry as the asymptotic attractor of topological-entropic evolution, unifying laboratory devices (indestructible pulsar, proton fusion catalyst) with inflationary past and dark-energy-dominated future.

\end{abstract}

\section{Introduction}

The TET--CVTL framework identifies the trefoil knot ($3_1$, linking number $L_k = 6$) as the unique primordial topological state. Multi-knot saturation progressively increases effective linking density toward Lk=100\%, achieving maximal entanglement entropy $S_{\text{ent}} = \ln 4$ per lattice cell.

This saturated state exhibits all hallmarks of de Sitter spacetime: constant positive curvature, conformal flatness, and exponential expansion. The emergence is parameter-free and derives directly from topological-entropic principles.

\section{De Sitter as the Saturated State of TET--CVTL}

In the Topological-Entropic Theory with Conformal Vacuum Tensor Lattice (TET--CVTL), the quantum vacuum is modeled as a conformal tensor lattice composed of primordial three-leaf clover (trefoil) knots with linking number \(L_k = 6\).

When this lattice reaches **multi-knot saturation at Lk=100\%** — as realized in the indestructible topological pulsar and the limiting Omega Point state — the entanglement entropy per lattice cell achieves its theoretical maximum:

\begin{equation}
    S_{\text{ent}} = \ln 4 \approx 1.3863 \quad \text{(per lattice cell)}
\end{equation}

This maximally symmetric saturated state exhibits:
\begin{itemize}
    \item Constant positive scalar curvature (from uniform topological saturation)
    \item Locally exponential expansion (driven by cosmic entropic dilution)
    \item Cosmological horizons (regions beyond which braiding coherence is inaccessible)
\end{itemize}

→ These are precisely the defining features of **de Sitter spacetime**.

The saturated TET--CVTL lattice therefore **emerges as de Sitter geometry** on cosmological scales.

\subsection{Emergent Cosmological Constant \(\Lambda\)}

As derived in the v2.0 treatment (DOI: 10.5281/zenodo.18150345), the cosmological constant \(\Lambda\) arises as an **entropic dilution effect** from the rarefaction of trefoil knots over cosmic time.

In the future remote, when ordinary matter and radiation are sufficiently diluted, only the residual topological vacuum energy remains — driving eternal accelerated expansion characteristic of an asymptotically de Sitter universe.

\subsection{Omega Point as Globally Saturated de Sitter}

The convergence toward the Omega Point (DOI: 10.5281/zenodo.18150631) represents the final global state:
\begin{itemize}
    \item Maximal complexity = maximal global linking saturation
    \item Unified consciousness = universal entanglement (\(S_{\text{ent}} \to \max\))
    \item Final geometry = pure de Sitter with cosmological horizon isolating the universe in eternal topological coherence
\end{itemize}

Thus, de Sitter spacetime is not an external assumption but the **natural geometric background of the Omega Point** in TET--CVTL.

\subsection{Conformal Symmetry and Vacuum Tensor Lattice}

The designation "Conformal Vacuum Tensor Lattice" is deeply appropriate:
- De Sitter spacetime is conformally flat (mappable to Minkowski space via conformal transformation).
- The TET--CVTL lattice is intrinsically conformal (invariant under local rescalings).
→ The vacuum structure is naturally compatible with de Sitter symmetry, and full saturation forces the emergence of de Sitter geometry.

\subsection{Rigorous Synthesis}

The TET--CVTL begins with the primordial trefoil knot (topological "singularity"), evolves through local saturation (indestructible pulsar, proton fusion catalyst), and converges cosmologically toward eternal de Sitter (emergent \(\Lambda\) + Omega Point).

De Sitter is not postulated — it is the **natural asymptotic attractor** of the topological-entropic bootstrap.

The primordial knot has woven the fabric of eternal expansion.


\section{Mathematical Derivation of de Sitter Geometry}

The saturated TET--CVTL lattice has constant scalar curvature $R > 0$ due to uniform knot density. The effective cosmological constant arises from entropic dilution:

\begin{equation}
    \Lambda_{\text{eff}} = \frac{3 \hbar c}{l_P^2} \cdot \frac{\Delta S_{\text{ent}}}{V}
\end{equation}

where $\Delta S_{\text{ent}} = \ln 4$ per Planck-volume cell, yielding

\begin{equation}
    H^2 = \frac{\Lambda_{\text{eff}}}{3} = \constant
\end{equation}

The metric in conformal coordinates becomes exactly de Sitter:

\begin{equation}
    ds^2 = a^2(\tau) \left( -d\tau^2 + d\vec{x}^2 \right), \quad a(\tau) \propto e^{H\tau}
\end{equation}

\subsection{Entropic Origin of Exponential Expansion}

Cosmological dilution reduces local knot density while preserving global linking on holographic boundaries. The resulting vacuum energy density

\begin{equation}
    \rho_{\text{vac}} = \frac{\Lambda c^2}{8\pi G} = \frac{3 c^4}{8\pi G} \cdot \frac{\ln 4}{V_{\text{horizon}}}
\end{equation}

drives eternal accelerated expansion — identical to observed dark energy domination.

\subsection{Conformal Vacuum Tensor Lattice Compatibility}

The "conformal" nature of the TET--CVTL lattice ensures invariance under Weyl rescalings, matching de Sitter's conformal flatness. The saturated state is the fixed point of topological bootstrap:

\begin{equation}
    \lim_{L_k \to 100\%} S_{\text{ent}}(V) = \ln 4 \cdot V \quad \Rightarrow \quad \text{de Sitter vacuum}
\end{equation}

\section{Implications for Cosmology and Laboratory Devices}

- **Inflationary past**: early universe high knot density → large effective $\Lambda$ → exponential expansion.
- **Dark energy future**: matter dilution → residual topological vacuum → de Sitter asymptote.
- **Laboratory microcosms**: BOOTTECH pulsar at Lk=100\% reproduces de Sitter horizon effects (eternal coherence, no dissipation).

\begin{figure}[H]
\centering
\includegraphics[width=0.8\textwidth]{de_sitter_penrose_diagram.jpg} % Replace with actual file
\caption{Penrose conformal diagram of de Sitter spacetime – the asymptotic geometry of saturated TET--CVTL.}
\end{figure}

\begin{figure}[H]
\centering
\includegraphics[width=0.6\textwidth]{trefoil_lattice_saturation.jpg} % Replace with actual file
\caption{Progression from primordial trefoil to saturated lattice emerging as de Sitter vacuum.}
\end{figure}


\section{De Sitter Emergence in the TET--CVTL Framework}

The fully saturated state of the Topological-Entropic Theory with Conformal Vacuum Tensor Lattice (TET--CVTL) naturally emerges as a de Sitter spacetime. The primordial three-leaf clover (trefoil) knot lattice, when taken to maximal multi-knot saturation (Lk=100\%), produces constant positive curvature, maximal entanglement entropy, and exponential expansion driven by an emergent cosmological constant $\Lambda$.

\subsection{Derivation of de Sitter Geometry from Knot Saturation}

The vacuum tensor lattice is composed of trefoil units with linking number $L_k = 6$. Multi-knot saturation increases effective linking density until the system reaches the degenerate ground state of the multi-knot Ising anyonic model:

\begin{equation}
    S_{\text{ent}} = \ln W_{Lk=100\%} = \ln 4 \approx 1.3863 \quad \text{(per lattice cell)}
\end{equation}

This maximal entropy corresponds to constant positive scalar curvature $R > 0$. The effective cosmological constant arises from entropic dilution across cosmic scales:

\begin{equation}
    \Lambda_{\text{eff}} = 3 H^2 = \frac{3 c^2 \Delta S_{\text{ent}}}{l_P^2 V_{\text{cell}}}
\end{equation}

yielding exponential expansion $a(t) \propto e^{Ht}$ with constant Hubble parameter $H$. In conformal coordinates, the metric becomes exactly de Sitter:

\begin{equation}
    ds^2 = a^2(\tau) (-d\tau^2 + d\vec{x}^2)
\end{equation}

The conformal structure of the lattice ensures Weyl invariance, matching de Sitter's conformal flatness.

\subsection{Deriving Inflationary Dynamics from TET--CVTL}

In the early universe, high knot density produces a large effective $\Lambda_{\text{infl}}$:

\begin{equation}
    \Lambda_{\text{infl}} \propto \rho_{\text{knot}} \cdot \ln 4 \gg \Lambda_0
\end{equation}

driving quasi-de Sitter inflation with nearly constant $H_{\text{infl}}$. Slow-roll parameters emerge naturally:

\begin{align}
    \epsilon &= -\frac{\dot{H}}{H^2} \approx \frac{\dot{\rho}_{\text{knot}}}{\rho_{\text{knot}} H} \ll 1 \\
    \eta &= \epsilon - \frac{\ddot{H}}{2\dot{H} H} \approx 0
\end{align}

as knot dilution is gradual. Inflation ends when knot density drops below a critical threshold, transitioning to radiation domination — a parameter-free exit mechanism.

The primordial power spectrum inherits topological fluctuations from trefoil braiding statistics, predicting slight deviations from perfect scale-invariance testable with future CMB missions.

\subsection{Comparison to Loop Quantum Cosmology (LQC)}

Loop Quantum Cosmology resolves the Big Bang singularity via discrete spacetime and bounce dynamics. Key parallels and distinctions with TET--CVTL:

\begin{itemize}
    \item \textbf{Singularity resolution}: LQC replaces singularity with bounce; TET--CVTL replaces it with primordial saturated trefoil lattice (topological "pre-Bang" state).
    \item \textbf{Discrete structure}: LQC uses spin networks; TET--CVTL uses conformal vacuum tensor lattice of trefoil knots.
    \item \textbf{Inflation}: LQC requires additional inflaton field; TET--CVTL derives quasi-de Sitter inflation directly from knot density without new fields.
    \item \textbf{Late-time acceleration}: LQC typically needs separate dark energy; TET--CVTL unifies inflation and current $\Lambda$ from the same entropic mechanism.
    \item \textbf{Predictions}: Both predict modifications to CMB power spectrum at large scales; TET--CVTL adds topological signatures in non-Gaussianity from anyonic statistics.
\end{itemize}

Thus, TET--CVTL offers a purely topological alternative to LQC, eliminating the need for quantum geometry operators while preserving bounce-like behaviour through eternal knot saturation.

\begin{figure}[H]
\centering
\begin{subfigure}{0.45\textwidth}
    \centering
    \includegraphics[width=\textwidth]{de_sitter_inflation_tet.png}
    \caption{TET--CVTL inflationary phase from high knot density.}
\end{subfigure}
\hfill
\begin{subfigure}{0.45\textwidth}
    \centering
    \includegraphics[width=\textwidth]{lqc_vs_tetcvtl.png}
    \caption{Comparison of cosmological evolution: LQC bounce vs TET--CVTL primordial saturation.}
\end{subfigure}
\caption{Inflationary dynamics and singularity resolution in TET--CVTL vs Loop Quantum Cosmology.}
\end{figure}


\section{From Primordial Trefoil Saturation to Exponential Cosmic Expansion}

This preprint demonstrates that the fully saturated state of the Topological-Entropic Theory with Conformal Vacuum Tensor Lattice (TET--CVTL) naturally emerges as a de Sitter spacetime. The primordial three-leaf clover (trefoil) knot lattice, when taken to maximal multi-knot saturation (Lk=100\%), produces constant positive curvature, maximal entanglement entropy, and exponential expansion driven by an emergent cosmological constant $\Lambda$. This provides a parameter-free derivation of de Sitter geometry as the asymptotic attractor of topological-entropic evolution, unifying laboratory devices with inflationary past and dark-energy-dominated future. Specific CMB predictions and comparison to string cosmology are presented.


The TET--CVTL framework identifies the trefoil knot ($3_1$, linking number $L_k = 6$) as the unique primordial topological state. Multi-knot saturation progressively increases effective linking density toward Lk=100\%, achieving maximal entanglement entropy $S_{\text{ent}} = \ln 4$ per lattice cell.

This saturated state exhibits all hallmarks of de Sitter spacetime: constant positive curvature, conformal flatness, and exponential expansion.

\section{Derivation of de Sitter Geometry from Knot Saturation}

The saturated lattice has constant scalar curvature $R > 0$ due to uniform knot density. The effective cosmological constant arises from entropic dilution:

\begin{equation}
    \Lambda_{\text{eff}} = 3 H^2 = \frac{3 c^2 \Delta S_{\text{ent}}}{l_P^2 V_{\text{cell}}}
\end{equation}

yielding exponential expansion $a(t) \propto e^{Ht}$ with constant Hubble parameter $H$. In conformal coordinates, the metric becomes exactly de Sitter.

\section{Deriving Inflationary Dynamics from TET--CVTL}

In the early universe, high knot density produces a large effective $\Lambda_{\text{infl}} \gg \Lambda_0$, driving quasi-de Sitter inflation with nearly constant $H_{\text{infl}}$.

Slow-roll parameters emerge naturally from gradual knot dilution:

\begin{align}
    \epsilon &= -\frac{\dot{H}}{H^2} \approx \frac{\dot{\rho}_{\text{knot}}}{\rho_{\text{knot}} H} \ll 1 \\
    \eta &= \epsilon - \frac{\ddot{H}}{2 \dot{H} H} \approx 0
\end{align}

Inflation ends when knot density drops below critical threshold, transitioning to radiation domination — a parameter-free exit mechanism.

\section{Specific CMB Predictions from TET--CVTL Inflation}

The primordial power spectrum inherits topological fluctuations from trefoil braiding statistics.

\begin{itemize}
    \item \textbf{Scalar spectral index}: $n_s - 1 = 2\eta - 6\epsilon \approx -2/N_e$ (standard slow-roll), but with small topological correction $\delta n_s \sim (\ln 4)/N_e^2 \approx 0.002$ for $N_e \approx 60$.
    \item \textbf{Tensor-to-scalar ratio}: $r = 16\epsilon \approx 0.01$–0.003, within reach of CMB-S4 and LiteBIRD.
    \item \textbf{Non-Gaussianity}: anyonic braiding produces characteristic equilateral bispectrum $f_{\text{NL}}^{\text{equil}} \sim \mathcal{O}(1)$ from three-knot vertices.
    \item \textbf{Large-scale suppression}: knot coherence length imprints cutoff at low $\ell \lesssim 30$, reducing power on largest scales (potential explanation for observed low-$\ell$ anomalies).
\end{itemize}

These predictions are falsifiable with next-generation CMB experiments.


\subsection{Comparison with Planck CMB Data and Low-$\ell$ Suppression in TET--CVTL}

The Planck satellite (2013–2018 releases) has revealed a notable suppression of power in the cosmic microwave background temperature anisotropy spectrum at large angular scales (low multipoles $\ell \lesssim 30$). This "low-$\ell$ anomaly" manifests as a deficit in the observed $C_\ell^{TT}$ compared to the best-fit $\Lambda$CDM model, with significance $\sim 2$--$3\sigma$ depending on the multipole range and foreground treatment.

In the TET--CVTL framework, this suppression arises naturally from the finite **coherence length** of primordial trefoil braiding during inflation. The characteristic scale of the trefoil knot lattice imprints a topological cutoff on super-horizon modes:

\begin{equation}
    \ell_{\text{coh}} \approx \frac{2\pi}{\theta_{\text{knot}}} \cdot \frac{a_{\text{end}}}{a_0} \sim 20\text{--}40
\end{equation}

where $\theta_{\text{knot}}$ is the angular size of the knot coherence volume at the end of inflation.

Modes with $\ell < \ell_{\text{coh}}$ experience reduced amplification due to topological protection — the anyonic braiding suppresses long-wavelength fluctuations while preserving near-scale-invariance at smaller scales.

Predicted features:
\begin{itemize}
    \item Power suppression factor $\Delta C_\ell / C_\ell^{\Lambda\text{CDM}} \approx -10\%$ to $-30\%$ for $2 \leq \ell \leq 30$
    \item Smooth transition to standard $\Lambda$CDM power law at higher $\ell$
    \item Mild enhancement of Sachs-Wolfe plateau coherence due to topological phase locking
\end{itemize}

These predictions align qualitatively and quantitatively with Planck 2018 low-$\ell$ data (particularly the TT spectrum after foreground cleaning) and offer a physical explanation rooted in primordial knot topology rather than statistical fluke or systematic error.


The low-$\ell$ suppression constitutes a distinctive, falsifiable signature of topological inflation in TET--CVTL, distinguishable from alternative explanations (e.g., cosmic variance, modified initial state, or topological defects) by its specific scale dependence and absence of corresponding features in polarization spectra beyond the expected reionization bump.

Future CMB experiments (CMB-S4, LiteBIRD) will test this prediction with increased precision at low multipoles, potentially confirming or ruling out the primordial knot imprint.

\section{Comparison to String Cosmology}

String cosmology (e.g., brane inflation, string gas cosmology) derives inflation from extra dimensions and fundamental strings.

Key parallels and distinctions with TET--CVTL:

\begin{itemize}
    \item \textbf{Fundamental objects}: String cosmology uses 1D strings and branes; TET--CVTL uses 3D topological knots in 4D vacuum lattice.
    \item \textbf{Inflation mechanism}: String models require moduli stabilization or brane motion; TET--CVTL derives quasi-de Sitter directly from knot density without extra fields.
    \item \textbf{CMB signatures}: String cosmology predicts cosmic strings and specific non-Gaussian patterns; TET--CVTL predicts anyonic equilateral non-Gaussianity and low-$\ell$ suppression from knot coherence.
    \item \textbf{Late-time acceleration}: String models often need separate dark energy; TET--CVTL unifies inflation and current $\Lambda$ from the same entropic dilution.
    \item \textbf{Parameter freedom}: String cosmology has many moduli; TET--CVTL is completely parameter-free, with only the primordial trefoil as input.
\end{itemize}

TET--CVTL thus offers a minimal, purely 4D topological alternative to string-theoretic inflation while preserving testable predictions.


\section{Comparisons with Alternative Cosmological Models}

The TET--CVTL framework offers a unique topological approach to de Sitter emergence, inflation, and late-time acceleration. This section compares it to leading alternative models, highlighting distinctions in mechanism, predictions, and theoretical consistency.

\subsection{Comparison to Loop Quantum Cosmology (LQC)}

Loop Quantum Cosmology replaces the classical Big Bang singularity with a quantum bounce driven by discrete spacetime area operators and repulsive quantum geometry effects at Planck densities.

Key distinctions:
\begin{itemize}
    \item Singularity resolution in LQC is dynamical (bounce from high curvature); in TET--CVTL it is topological — the primordial saturated trefoil lattice provides a stable "pre-Bang" state with maximal entanglement entropy $\ln 4$.
    \item LQC requires quantization of geometric operators and the Barbero-Immirzi parameter; TET--CVTL is parameter-free, relying solely on the linking number of the primordial knot.
    \item Inflation in LQC typically requires a subsequent inflaton field; TET--CVTL derives quasi-de Sitter inflation directly from high primordial knot density without additional fields.
    \item Late-time acceleration in LQC needs separate dark energy; TET--CVTL unifies early inflation and current $\Lambda$ through the same entropic dilution mechanism.
    \item CMB predictions in LQC include potential large-scale power suppression from pre-bounce effects; TET--CVTL predicts low-$\ell$ suppression from knot coherence length plus anyonic equilateral non-Gaussianity.
\end{itemize}

TET--CVTL provides a complementary, purely topological alternative to LQC: singularity resolved by eternal knot saturation rather than quantum geometry, with unified description of bounce-like behaviour, inflation, and late acceleration.

\subsection{Comparison to Brane-World Cosmology}

Brane-world scenarios embed our 4D universe as a brane in a higher-dimensional bulk, with gravity propagating into extra dimensions.

Key distinctions:
\begin{itemize}
    \item TET--CVTL is purely 4D, with effective "extra-dimensional" behaviour emerging from topological saturation of the conformal vacuum tensor lattice.
    \item Gravity modification in brane-world arises from warped geometry; in TET--CVTL, emergent $G$ and $\Lambda$ derive from local knot saturation and cosmic entropic dilution without extra dimensions.
    \item Inflation in brane-world requires radion stabilization or brane motion; TET--CVTL derives quasi-de Sitter inflation directly from primordial knot density.
\end{itemize}

TET--CVTL achieves similar phenomenological goals in a minimal 4D setting.

\subsection{Comparison to Dvali-Gabadadze-Porrati (DGP) Model}

The DGP model induces late-time self-acceleration through gravity leakage into a 5D Minkowski bulk via a crossover scale $r_c$.

Key distinctions:
\begin{itemize}
    \item DGP modifies the Friedmann equation at large scales; TET--CVTL derives acceleration from entropic dilution of saturated knots.
    \item The self-accelerating DGP branch suffers from ghost instabilities (negative kinetic term for the helicity-0 graviton mode).
    \item TET--CVTL is ghost-free: all degrees of freedom have positive norm due to topological protection and positive-definite entanglement entropy.
\end{itemize}

TET--CVTL reproduces late-time acceleration without ghosts or extra dimensions.

\subsection{Ghost Instabilities in DGP Self-Accelerating Branch}

The self-accelerating DGP solution exhibits a ghost mode with negative kinetic energy, leading to exponential growth of instabilities and violation of unitarity. This pathology is unavoidable in the accelerating branch and renders the model inconsistent as an effective field theory.

TET--CVTL avoids such issues entirely: stability is guaranteed by global topological invariants and the positive-definite nature of entanglement entropy in the saturated lattice.

\subsection{Comparison to Ekpyrotic/Cyclic Cosmology}

Ekpyrotic models propose a cyclic universe with brane collisions replacing the Big Bang, generating scale-invariant perturbations during a contracting phase.

Key distinctions:
\begin{itemize}
    \item Ekpyrotic requires extra dimensions and brane dynamics; TET--CVTL is 4D and single-shot.
    \item Perturbations in ekpyrotic arise during contraction; in TET--CVTL during quasi-de Sitter expansion from high knot density.
    \item Ekpyrotic is cyclic; TET--CVTL converges monotonically to de Sitter Omega Point.
\end{itemize}

TET--CVTL provides a simpler, non-cyclic alternative with unified description of early and late acceleration.

\subsection{Anyonic Statistics and CMB Predictions}

Primordial trefoil braiding induces Ising-type anyonic statistics with phase $\theta = 6\pi/5$. These anyons generate distinctive equilateral non-Gaussianity and mild scale dependence in the primordial bispectrum, testable with future CMB experiments.

\subsection{Comparison to Hořava-Lifshitz Gravity}

Hořava-Lifshitz gravity breaks Lorentz invariance at high energies via anisotropic scaling ($z=3$) to achieve power-counting renormalizability.

Key distinctions:
\begin{itemize}
    \item Hořava-Lifshitz violates Lorentz symmetry; TET--CVTL preserves full Lorentz invariance and conformal symmetry.
    \item UV improvement in Hořava-Lifshitz comes from higher spatial derivatives; in TET--CVTL from discrete topological saturation acting as natural regulator.
    \item Early versions of Hořava-Lifshitz suffered from ghosts and strong coupling; TET--CVTL is ghost-free by construction.
\end{itemize}

TET--CVTL achieves UV regularity and cosmological predictions while maintaining Lorentz invariance.

\subsection{Equilateral Non-Gaussianity}

The dominant non-Gaussianity in TET--CVTL is equilateral, with predicted amplitude $f_{\text{NL}}^{\text{equil}} \sim 3$--$8$ from three-knot anyonic vertices.

\subsection{Scale Dependence of Non-Gaussianity}

Running arises from knot dilution during inflation:
\begin{equation}
    n_{f_{\text{NL}}} \approx -0.01 \text{ to } -0.03
\end{equation}

\subsection{Comparison to Orthogonal Non-Gaussianity}

Orthogonal shape exhibits sign oscillations and arises from higher-derivative kinetic terms. TET--CVTL equilateral shape is purely positive with topological origin, offering clear distinguishability.

\subsection{Parity-Odd Contributions}

Chiral trefoil statistics induce parity-odd TB and EB correlations with amplitude $\sim 0.1$--$1 \, \mu$K$^2$ at $\ell \approx 200$, a smoking-gun signature absent in parity-even models.

\subsection{Comparison to Flattened Non-Gaussianity}

Flattened shape peaks in folded configurations from non-Bunch-Davies vacua. TET--CVTL equilateral shape has minimal folded contribution and additional parity-odd signatures.

\subsection{LiteBIRD Sensitivity}

LiteBIRD forecasts $\sigma(f_{\text{NL}}^{\text{equil}}) \approx 3.5$ (standalone), sufficient for $>2\sigma$ detection of TET--CVTL central value.

\subsection{Comparison to Local Non-Gaussianity}

Local shape peaks in squeezed limit from multi-field evolution. TET--CVTL equilateral has negligible squeezed contribution and distinct topological origin.

\subsection{CMB-S4 Sensitivity}

CMB-S4 forecasts $\sigma(f_{\text{NL}}^{\text{equil}}) \approx 3.0$--$4.5$, enabling high-significance discrimination of equilateral template.

\subsection{Comparison to Orthogonal Non-Gaussianity (Revisited)}

Orthogonal shape's oscillatory nature and kinetic origin contrast sharply with TET--CVTL's positive equilateral shape from protected anyonic vertices.

\end{section}


\section{Conclusions}

The TET--CVTL framework, through progressive multi-knot saturation of the primordial three-leaf clover (trefoil) knot, naturally converges to a de Sitter spacetime as its asymptotic state. The maximal linking saturation (Lk=100\%) produces constant positive curvature, maximal entanglement entropy per lattice cell $S_{\text{ent}} = \ln 4$, and exponential expansion driven by an emergent cosmological constant derived from entropic dilution. This emergence is parameter-free: de Sitter geometry is not postulated but arises directly from the topological bootstrap, unifying the inflationary past (high knot density) with the dark-energy-dominated future (residual vacuum energy).

The same mechanism manifests in laboratory-scale devices: the indestructible topological pulsar achieves eternal coherence and perfect pulsation through global topological protection, while the proton fusion catalyst exploits the primordial anyonic phase $\theta = 6\pi/5$ to enhance fusion rates by 20--40$\times$ in ultraclean environments. The rigorous QED-derived upper bound on macroscopic vacuum torque closes speculative propulsion schemes, providing theoretical closure alongside constructive open designs.

Comparisons with alternative cosmologies — brane-world models, DGP self-acceleration, ekpyrotic cycles, Hořava-Lifshitz gravity, loop quantum cosmology, and string-theoretic approaches — highlight the distinctive strengths of TET--CVTL: full Lorentz invariance, absence of ghosts or instabilities, no extra dimensions, complete parameter freedom (sole input: the primordial trefoil knot), and unified description of singularity resolution, inflation, late acceleration, and testable CMB signatures (low-$\ell$ suppression, equilateral non-Gaussianity with mild negative running, parity-odd TB/EB correlations).

These signatures — particularly the equilateral non-Gaussianity $f_{\text{NL}}^{\text{equil}} \sim 3$--$8$ and chiral parity violation — lie within reach of upcoming experiments (CMB-S4, LiteBIRD), offering clear falsifiability and potential smoking-gun evidence for topological inflation.

Ultimately, the TET--CVTL framework reveals a profound unity: the same primordial knot that weaves eternal laboratory order also threads the fabric of cosmic evolution, from quasi-de Sitter inflation to the final de Sitter Omega Point. The bootstrap loop is closed — the knot has woven the universe.

All positive designs and codes are released under Creative Commons Attribution-NonCommercial 4.0 International (CC BY-NC 4.0) for open non-commercial research and replication.



\section{Acknowledgments}

This work represents an ongoing independent exploration within the TET Collective framework. The author gratefully acknowledges the invaluable collaborative partnership with Grok (xAI), whose insightful discussions, rigorous critical analysis, and creative contributions have been instrumental throughout the development of the TET--CVTL theory — from primordial knot selection to de Sitter emergence and laboratory applications.

Special thanks are due to the open-source scientific community, particularly the developers of QuTiP, NumPy, Matplotlib, and LaTeX packages, whose tools enabled the replicable simulations and clear presentation of results.

The author expresses deep appreciation to the broader community of independent researchers and thinkers who continue to challenge conventional paradigms and pursue parameter-free unification from first principles.

Finally, this preprint is dedicated to the primordial trefoil knot — the simplest non-trivial topological invariant that weaves eternal order from the conformal vacuum.



\section{License}

This work, including all text, derivations, equations, simulation codes, figures, and diagrams, is licensed under a \textbf{Creative Commons Attribution-NonCommercial 4.0 International License (CC BY-NC 4.0)}.

You are free to:
\begin{itemize}
    \item \textbf{Share} — copy and redistribute the material in any medium or format
    \item \textbf{Adapt} — remix, transform, and build upon the material
\end{itemize}

Under the following terms:
\begin{itemize}
    \item \textbf{Attribution} — You must give appropriate credit to the author (Simon Soliman, TET Collective, Rome, Italy), provide a link to the license, and indicate if changes were made. You may do so in any reasonable manner, but not in any way that suggests the licensor endorses you or your use.
    \item \textbf{NonCommercial} — You may not use the material for commercial purposes.
\end{itemize}

No additional restrictions apply. Derivative works may be distributed under different terms, provided they respect the NonCommercial condition.

Full license text available at: \url{https://creativecommons.org/licenses/by-nc/4.0/legalcode}

This preprint is the result of independent research by the TET Collective.

\section{Bibliography}

\begin{thebibliography}{9}

\bibitem{Planck2018}
Planck Collaboration (2018). 
Planck 2018 results. VI. Cosmological parameters. 
\textit{Astron. Astrophys.} \textbf{641}, A6.
\url{https://doi.org/10.1051/0004-6361/201833910}

\bibitem{CMB-S4}
CMB-S4 Collaboration (2022). 
Snowmass 2021 CMB-S4 white paper. 
arXiv:2203.07638.

\bibitem{LiteBIRD}
LiteBIRD Collaboration (2023). 
LiteBIRD: mission overview and design. 
\textit{J. Low Temp. Phys.} \textbf{212}, 428.

\bibitem{Horava2009}
Hořava, P. (2009). 
Quantum gravity at a Lifshitz point. 
\textit{Phys. Rev. D} \textbf{79}, 084008.

\bibitem{DGP2000}
Dvali, G., Gabadadze, G., \& Porrati, M. (2000). 
4D gravity on a brane in 5D Minkowski space. 
\textit{Phys. Lett. B} \textbf{485}, 208.

\bibitem{Ekpyrotic2001}
Steinhardt, P. J., \& Turok, N. (2002). 
Cosmic evolution in a cyclic universe. 
\textit{Phys. Rev. D} \textbf{65}, 126003.

\bibitem{LQC}
Ashtekar, A., \& Singh, P. (2011). 
Loop quantum cosmology: a status report. 
\textit{Class. Quant. Grav.} \textbf{28}, 213001.

\bibitem{StringCosmo}
Brandenberger, R. (2017). 
String gas cosmology: progress and problems. 
\textit{Class. Quant. Grav.} \textbf{34}, 094001.

\bibitem{RS1999}
Randall, L., \& Sundrum, R. (1999). 
A large mass hierarchy from a small extra dimension. 
\textit{Phys. Rev. Lett.} \textbf{83}, 3370.

\end{thebibliography}


\end{document}